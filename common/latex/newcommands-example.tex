%% new commands
%\newfloat{Video}{tbp}{}
%\def\ext{avi}
\newcommand{\citenew}[1]{\textit{\scriptsize\textcolor{blue}{\cite{#1}}}}
\newcommand{\partialval}[1]{\ifes@solutions (\textbf{#1}) \fi} % for esami extra string
\newcommand{\bbb}[1]{\begin{block}{#1}}
  \newcommand{\beb}{\end{block}}
%%\newcommand{\bbb}[1]{\begin{beamerboxesrounded}{#1}}
%%  \newcommand{\beb}{\end{beamerboxesrounded}}
\newcommand{\bbc}{\begin{columns}}
  \newcommand{\bec}{\end{columns}}
\newcommand{\be}{\begin{equation}}
  \newcommand{\ee}{\end{equation}}
\newcommand{\bi}{\begin{itemize}}
  \newcommand{\ei}{\end{itemize}}
\newcommand{\bft}{\frametitle}
\newcommand{\bfst}{\framesubtitle}
\newcommand{\fig}[2]{\begin{center}\includegraphics[width=#1\columnwidth]{#2}\end{center}}
\newcommand{\deriv}[2]{\ensuremath{\frac{\partial #1}{\partial #2}}}
\newcommand{\derivn}[3]{\ensuremath{\frac{\partial^#3 #1}{\partial #2^#3}}}
\newcommand{\parcial}[2] {\frac{\partial{#1}}{\partial{#2}}}
\newcommand{\parcialdos}[3] {\frac{\partial^2{#1}}{\partial{#2}\partial{#3}}}
\newcommand{\parcialcons}[3] {\left(\frac{\partial{#1}}{\partial{#2}}\right)_{#3}}
\newcommand{\parcialconsgen} [4]{\left(\frac{\partial^{#4}{#1}}{\partial{#2}^{#4}}\right)_{#3}}
\newcommand{\jacob}[4] {\frac{\partial\left({#1},{#2}\right)}{\partial\left({#3},{#4}\right)}}
\newcommand{\Abs}[1]{\ensuremath{\bigl\lvert #1 \bigr\rvert}}
%\newcommand{\Abs}[1] {\mid #1 \mid}
\newcommand{\A}{\ensuremath{A}}
\newcommand{\Q}{\ensuremath{\mathbf{q}}}
\newcommand{\qcero}{\ensuremath{q_0}}
\newcommand{\quno}{\ensuremath{q_1}}
\newcommand{\qdos}{\ensuremath{q_2}}
\newcommand{\qtres}{\ensuremath{q_3}}
\newcommand{\aij}{\ensuremath{a_{{ij}}}}
\newcommand{\hij}{\ensuremath{h_{{ij}}}}
\newcommand{\fijt}{\ensuremath{f_{{ij}^{t}}}}
\newcommand{\fjit}{\ensuremath{f_{{ji}^{t}}}}
\newcommand{\nij}{\ensuremath{\hat{n}_{{ij}\,}}}
\newcommand{\tij}{\ensuremath{\hat{t}_{{ij}\,}}}
\newcommand{\tji}{\ensuremath{\hat{t}_{{ji}\,}}}
\newcommand{\ri}{\ensuremath{\vec{r}_{{i}}}}
\newcommand{\rj}{\ensuremath{\vec{r}_{{j}}}}
\newcommand{\vt}{\ensuremath{\vec{v}_{{t}}}}
\newcommand{\vi}{\ensuremath{\vec{v}_{{i}}}}
\newcommand{\vj}{\ensuremath{\vec{v}_{{j}}}}
\newcommand{\vij}{\ensuremath{\vec{v}_{{ij}}}}
\newcommand{\vpi}{\ensuremath{\vec{v_p}_{i}}}
\newcommand{\vpj}{\ensuremath{\vec{v_p}_{j}}}
\newcommand{\wi}{\ensuremath{\vec{\omega}_{i}}}
\newcommand{\wj}{\ensuremath{\vec{\omega}_{j}}}
\newcommand{\vpij}{\ensuremath{\vec{v_p}_{ij}}}
\newcommand{\vpji}{\ensuremath{\vec{v_p}_{ji}}}
\newcommand{\tr}{\ensuremath{\theta_r}}
\newcommand{\hs}{\ensuremath{h_s}\,}
\newcommand{\n}{\ensuremath{\eta}\,}
\newcommand{\mus}{\ensuremath{\mu_s}}
\newcommand{\mud}{\ensuremath{\mu_d}}
\newcommand{\vxi}{\ensuremath{\vec{\xi}\,}}
\newcommand{\ftemp}{\ensuremath{\vec{f}_t^0\,}}
\newcommand{\mij}{\ensuremath{m_{ij}}}
\newcommand{\fn}{\ensuremath{\vec{f}_{n}}}
\newcommand{\ft}{\ensuremath{\vec{f}_{t}}}
\newcommand{\en}{\ensuremath{\epsilon_n}}
\newcommand{\eq}{\ensuremath{\epsilon_q}}
\newcommand{\ep}{\ensuremath{\epsilon_p}}
\newcommand{\N}{\ensuremath{\mathcal{N}}}
\newcommand{\W}{\ensuremath{\mathcal{W}}}
\newcommand{\FH}{\ensuremath{\overrightarrow{FH}}}
\newcommand{\FDN}{\ensuremath{\overrightarrow{F_{DAMP}}}}
\newcommand{\FDYN}{\ensuremath{\overrightarrow{F_{DYN}}}}
\newcommand{\FROL}{\ensuremath{\overrightarrow{F_{ROL}}}}
\newcommand{\FSTA}{\ensuremath{\overrightarrow{F_{STA}}}}
\newcommand{\Fijn}{\ensuremath{\Vec{F}_{ij}^n}}
\newcommand{\Fjin}{\ensuremath{\Vec{F}_{ji}^n}}
\newcommand{\la}{\ensuremath{\langle}}
\newcommand{\ra}{\ensuremath{\rangle}}
\newcommand{\mean}[1]{\ensuremath{\langle #1 \rangle}}
\newcommand{\defgraphwidth}{0.7\textwidth}
\newcommand{\dt}{\ensuremath{\delta t}}
\newcommand{\X}{\chi}
\newcommand{\sw}{\sigma_{\rm wall}}
\newcommand{\vw}{v_{\rm wall}}
\newcommand{\pmax}{\phi_{\rm max}}
\newcommand{\rmax}{r_{\rm max}}
\newcommand{\rmin}{r_{\rm min}}
\newcommand{\vmin}{v_{\rm min}}
\newcommand{\Vmin}{V_{\rm min}}
\newcommand{\tmax}{t_{\rm max}} 
\newcommand{\apone}{\ensuremath{\alpha + 1}}
\newcommand{\ua}{\ensuremath{1/\alpha}}
\newcommand{\dmax}{\ensuremath{d_{\textrm{max}}}}
\newcommand{\dmin}{\ensuremath{d_{\textrm{min}}}}
\newcommand{\email}[1]{\href{mailto:#1}{#1}}
%% New colors
\definecolor{cafe}{cmyk}{0,0.81,1,0.60}
\definecolor{violeta}{cmyk}{0.07,0.90,0,0.34}
\definecolor{esmeralda}{cmyk}{0.91,0,0.88,0.12}
\definecolor{canela}{cmyk}{0.14,0.42,0.56,0}
\definecolor{marron}{cmyk}{0,0.72,1,0.45}
\definecolor{pino}{cmyk}{0.92,0,0.59,0.25}
\definecolor{aguamarina}{cmyk}{0.85,0,0.33,0}
\definecolor{gris-claro}{cmyk}{0,0,0,0.30}
\definecolor{gris-oscuro}{cmyk}{0,0,0,0.50}

%tables
\newcolumntype{M}{>{$}c<{$}}

%%%%%%%%%%%%%%%%%%%%%%%%%%%% 
%% math help commands
%%%%%%%%%%%%%%%%%%%%%%%%%%%% 
\newcommand{\abs}[1]{\ensuremath{\left\lvert #1 \right\rvert}}
%%%%%%%%%%%%%%%%%%%%%%%%%%%% 
%% vector printing
%%%%%%%%%%%%%%%%%%%%%%%%%%%% 
%\newcommand{\vecto}[1]{\ensuremath{\vec{\boldsymbol{#1}}}}
\newcommand{\vecto}[1]{\ensuremath{\overrightarrow{\boldsymbol{#1}}}}
%\newcommand{\Vecto}[1]{\ensuremath{\Vec{\boldsymbol{#1}}}}
\newcommand{\Vecto}[1]{\ensuremath{\overrightarrow{\boldsymbol{#1}}}}
\newcommand{\unit}[1]{\ensuremath{\widehat{\boldsymbol{#1}}}}
\newcommand{\Unit}[1]{\ensuremath{\widehat{\boldsymbol{#1}}}}
\newcommand{\hati}{\ensuremath{\widehat{\boldsymbol{\imath}}}}
\newcommand{\hatj}{\ensuremath{\widehat{\boldsymbol{\jmath}}}}
\newcommand{\hatk}{\ensuremath{\widehat{\boldsymbol{k}}}}
%%%%%%%%%%%
\newcommand{\suminverse}[2]{\dfrac{1}{#1_{\textrm{equiv}}^{\textrm{#2}}} = \sum \dfrac{1}{#1_i}}
\newcommand{\sumdirect}[2]{#1_{\textrm{equiv}}^{\textrm{#2}} = \sum #1_i}
\newcommand{\cross}[2]{\ensuremath{\vecto{#1} \times \vecto{#2}}}
\newcommand{\Cross}[2]{\ensuremath{\Vecto{#1} \times \Vecto{#2}}}
\newcommand{\DotPr}[2]{\ensuremath{\Vecto{#1} \circ \Vecto{#2}}}
%% \newcommand{\Abs}[1]{\ensuremath{\lvert #1 \rvert}}
\newcommand{\VF}{\Vecto{F}}
\newcommand{\VE}{\Vecto{E}}
\newcommand{\VB}{\Vecto{B}}
\newcommand{\Ur}{\unit{r}}
\newcommand{\DV}{\Delta V}


%%%%%%%%%%%%%%%%%%%%%%%%%%%%
%\newcommand{\ms}{\scriptscriptstyle}
\newcommand{\ms}[1]{\scalebox{0.4}{$#1$}}
\newcommand{\sss}[1]{\!_\ms{#1}}
%%%%%% TODO 
\newcommand{\xtodo}[2][]{\tikzexternaldisable\pgfkeys{/pgf/fpu=false}\todo[#1]{#2}\tikzexternalenable\pgfkeys{/pgf/fpu=true}}
\newcommand{\xmissingfigure}[2][]{\tikzexternaldisable\pgfkeys{/pgf/fpu=false}\missingfigure[#1]{#2}\tikzexternalenable\pgfkeys{/pgf/fpu=true}}


\newcommand{\whitebox}[1]{\fbox{{\textcolor{white}{#1}}}} 
%%%%%%%%%%%%%%%%%%%%%%%%%%%% 
%% Formatted numeric printing
%%%%%%%%%%%%%%%%%%%%%%%%%%%% 
%% pgf
% \newcommand{\Num}[1]{\pgfmathprintnumber{#1}}
%\newcommand{\NumPre}[2]{\pgfkeys{/pgf/number format/.cd, precision=#2}\pgfmathprintnumber{#1}\pgfkeys{/pgf/number format/.cd, precision=2}}
%\newcommand{\NumSign}[1]{\pgfkeys{/pgf/number format/showpos=true}\Num{#1}\pgfkeys{/pgf/number format/showpos=false}}
%\newcommand{\NumEval}[1]{\pgfmathparse{(1*#1)}\Num{\pgfmathresult}}
%\newcommand{\NumSI}[2]{\Num{#1}\ \si{#2}}
% %% Siunitx
% \newcommand{\Num}[1]{\pgfmathprintnumberto{#1}{\tmpnum}\ensuremath{\num{\tmpnum}}}
% \newcommand{\NumPre}[2]{\pgfmathprintnumberto{#1}{\tmpnum}\ensuremath{\num[round-mode=places,round-precision=#2]{\tmpnum}}}
% \newcommand{\NumFig}[2]{\pgfmathprintnumberto{#1}{\tmpnum}\ensuremath{\num[round-mode=figures,round-precision=#2]{\tmpnum}}}
% \newcommand{\NumSign}[1]{\pgfmathprintnumberto{#1}{\tmpnum}\ensuremath{\num[explicit-sign=+]{\tmpnum}}}
% \newcommand{\NumEval}[1]{\pgfmathparse{(1*(#1))}\Num{\pgfmathresult}}
% \newcommand{\NumEvalPre}[2]{\pgfmathparse{(1*(#1))}\NumPre{\pgfmathresult}{#2}}
% \newcommand{\NumEvalFig}[2]{\pgfmathparse{(1*(#1))}\NumFig{\pgfmathresult}{#2}}
% \newcommand{\NumSI}[2]{\mbox{\Num{#1}\ \si{#2}}}
%% siunitx + exams
%\newcommand{\Num}[1]{\pgfmathprintnumberto{#1}{\tmpnum}\ensuremath{\num[round-mode=figures,round-precision=3,scientific-notation = engineering]{\tmpnum}}}
\newcommand{\Num}[1]{\pgfmathprintnumberto{#1}{\tmpnum}\ensuremath{\num[round-mode=figures,round-precision=3,scientific-notation = engineering]{\tmpnum}}}
\newcommand{\NumPre}[2]{\pgfmathprintnumberto{#1}{\tmpnum}\ensuremath{\num[round-mode=places,round-precision=#2]{\tmpnum}}}
\newcommand{\NumFig}[2]{\pgfmathprintnumberto{#1}{\tmpnum}\ensuremath{\num[round-mode=figures,round-precision=#2]{\tmpnum}}}
\newcommand{\NumSign}[1]{\pgfmathprintnumberto{#1}{\tmpnum}\ensuremath{\num[explicit-sign=+]{\tmpnum}}}
\newcommand{\NumEval}[1]{\pgfmathparse{(1.0*(#1))}\Num{\pgfmathresult}}
\newcommand{\NumEvalPre}[2]{\pgfmathparse{(1*(#1))}\NumPre{\pgfmathresult}{#2}}
\newcommand{\NumEvalFig}[2]{\pgfmathparse{(1*(#1))}\NumFig{\pgfmathresult}{#2}}
\newcommand{\NumSI}[2]{\mbox{\Num{#1}\ \si{#2}}}



\newcommand{\phasors}[6]{
  % arg 1: SXC = XC/R
  % arg 2: ALPHA, XL = \ALPHA*\XC
  % arg 3: \L, limits for coordinates axis
  % arg 4: Angle for VC
  % arg 5: Angle for VL
  % arg 6: Label
  \pgfkeys{/pgf/fpu=false}
  \pgfmathsetmacro{\L}{\pgfmathfloatvalueof{#3}}
  \pgfmathsetmacro{\SXC}{\pgfmathfloatvalueof{#1}}
  \pgfmathsetmacro{\ALPHA}{\pgfmathfloatvalueof{#2}}
  \pgfmathsetmacro{\SXL}{\pgfmathfloatvalueof{\ALPHA*\ALPHA*\SXC}}
  \begin{tikzpicture}
    % \begin{axis}[scale=0.35, axis equal=true, scale only axis, grid=both, xmin=\XMIN, xmax=\XMAX, ymin=\YMIN, ymax=\YMAX, xticklabel=\empty, yticklabel=\empty, minor tick num=1, axis lines = middle, xlabel={$x$}, ylabel={$y$}, label style={at={(ticklabel cs:1.1)}}]
    \begin{axis}[scale=0.35, axis equal=true, scale only axis, grid=both, xmin=-\L, xmax=\L, ymin=-\L, ymax=\L, xticklabel=\empty, yticklabel=\empty, minor tick num=1, axis lines = middle, xlabel={$x$}, ylabel={$y$}, label style={at={(ticklabel cs:1.1)}}]
      \coordinate (A) at (0, 0);
      \coordinate (VR) at (0.707, 0.707);
      \coordinate (VC) at ($(A)!\SXC!#4:(VR)$);
      \coordinate (VL) at ($(A)!\SXL!#5:(VR)$);
      \draw [->,line width=2pt, red] (A) -- (VR) node[above,scale=1.3, pos=0.8] {$v_R$};
      \draw [->,line width=2pt, orange, dashed] (A) -- (VC)
      node[right,scale=1.3, pos=0.8] {$v_C$};
      \draw [->,line width=2pt, olive, dotted] (A) -- (VL)
      node[above,scale=1.3, pos=0.8] {$v_L$};
      \node at (-0.8*\L, -0.8*\L) [scale=1.5] {\textbf{(#6)}};
    \end{axis}
  \end{tikzpicture}
  \pgfkeys{/pgf/fpu=true}
}

%%%%%%%%%%%%%%%%%%%%%%%%%%%% 
% enviroments
\newenvironment{problem}
{{}{}}
%\newenvironment{problem*}
%{{}{}}

%%% boxes for exams
\newcommand{\solbox}[1][5]{\pgfkeys{/pgf/fpu=false}\begin{tcolorbox}[beforeafter skip=0pt,colback=yellow!2,width=0.95\linewidth, height=#1 ex]\end{tcolorbox}\pgfkeys{/pgf/fpu=true}}
%\newcommand{\solbox}[1][5]{\framebox{\rule{0pt}{#1ex}}}
\newenvironment{solution}
    {\pgfkeys{/pgf/fpu=false}
      \begin{tcolorbox}[parbox,title={solution},beforeafter skip=1pt,colback=green!5,width=0.95\linewidth]
    }
    {
    \end{tcolorbox}
    \pgfkeys{/pgf/fpu=true}
    }
%tables
\newcolumntype{M}{>{$}c<{$}}
