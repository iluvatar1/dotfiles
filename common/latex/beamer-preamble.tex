%% PACKAGES
\usepackage[spanish,english]{babel}
% or whatever
\usepackage[latin1]{inputenc}
%\usepackage{times}
%\usepackage[T1]{fontenc}
\usepackage{ae}
\usepackage{aecompl}
%\renewcommand\mathfamilydefault{\rmdefault}
%\renewcommand{\rmdefault}{cmss}
\usefonttheme[onlymath]{serif}
\usepackage{multimedia}
\usepackage{media9}
\usepackage{fancybox}
\usepackage{algorithmic}
\usepackage{array}
\usepackage{booktabs}
\usepackage{array}
\usepackage{framed} % for shaded environments
\usepackage[scientific-notation=true]{siunitx}
\usepackage{tikz}
\usetikzlibrary{mindmap}
%\hypersetup{backref, pdfpagemode=FullScreen, colorlinks=true,  linkcolor=gray}
\hypersetup{backref, colorlinks=true,  linkcolor=gray}

%% from https://www.sharelatex.com/blog/2013/08/29/tikz-series-pt3.html
\usetikzlibrary{shapes.geometric, arrows}
\tikzstyle{startstop} = [rectangle, rounded corners, minimum width=3cm, minimum height=1cm,text centered, draw=black, fill=red!30]
\tikzstyle{io} = [trapezium, trapezium left angle=70, trapezium right angle=110, minimum width=3cm, minimum height=1cm, text centered, draw=black, fill=blue!30]
\tikzstyle{process} = [rectangle, minimum width=3cm, minimum height=1cm, text centered, draw=black, fill=orange!30]
\tikzstyle{decision} = [diamond, minimum width=3cm, minimum height=1cm, text centered, draw=black, fill=green!30]
\tikzstyle{arrow} = [thick,->,>=stealth]

%% Beamer setup
\mode<presentation>
{
  %% \usetheme{Warsaw}
  %% \usetheme{Boadilla}
  \usetheme{Darmstadt}
  % \usetheme{default}
  %% \usetheme{CambridgeUS}
  %% \usetheme{Rochester}
  %% \usetheme{Singapore} %% punticos, light color
  %%\usetheme{Berlin} %% punticos, bad color
  %% or ...
  
  %% \setbeamercovered{transparent}
  %% or whatever (possibly just delete it)
  
  %% \useinnertheme[shadow]{rounded}
  
  %% \usecolortheme{albatross}
  %% \usecolortheme{beaver}
  \usecolortheme[named=Brown]{structure}
  %% \usecolortheme[RGB={205,173,0}]{structure}
  
  %\usefonttheme[onlysmall]{structurebold}
  %\usefonttheme[onlymath]{serif}
  %\usefonttheme{professionalfonts}
}

%% Delete this, if you do not want the table of contents to pop up at
%% the beginning of each subsection:
%%\AtBeginSubsection[]
% \AtBeginSection[]
% {
%   \begin{frame}<beamer>
%     \frametitle{Outline}
%     \tableofcontents[currentsection,currentsubsection]
%   \end{frame}
% }

%\AtBeginSubsection[]
% {
%   \begin{frame}<beamer>
%     \frametitle{Outline}
%     \tableofcontents[currentsection,currentsubsection]
%   \end{frame}
% }
% If you wish to uncover everything in a step-wise fashion, uncomment
% the following command: 
%\beamerdefaultoverlayspecification{<+->}
