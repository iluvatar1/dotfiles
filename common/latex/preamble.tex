% Packages
% Test graphicx since I am not using it here, why??
%\usepackage[usenames,dvipsnames,svgnames,table]{xcolor}
\usepackage{xcolor}
\usepackage{comment} % for org-latex tables
\usepackage{dcolumn} % to align tables at decimal point, use D column type
\usepackage{cite} % improves handling of numerical citations
\usepackage{fancyvrb} % verbatim with line number, input and output
\usepackage{verbatim}
\usepackage{etex} % To solve problems with new dims and pgf/tkiz
\usepackage{lmodern} % superseeds ae and aecompl
\renewcommand{\rmdefault}{cmss}
\usepackage{framed} % for shaded environments
%\definecolor{shadecolor}{rgb}{0,0,0}
%\colorlet{shadecolor}{lightgray!40}
\definecolor{shadecolor}{rgb}{0.686,0.933,0.933}
\usepackage{etoolbox} % To use modern ifdef
%\usepackage[latin1,utf8]{inputenc}
%\usepackage[utf8]{inputenc}
\usepackage[latin1,utf8]{inputenc}
%\inputencoding{latin1}
\usepackage{amsmath,amstext,amssymb}
\usepackage{tikz} % also for pgfmathparse
\usepackage{pgfplots} % also for plotting
\usepackage{pgfmath} % also for pgfmathparse
\usepackage[scientific-notation=true]{siunitx}
\usepackage{lipsum} % for dummy text, use as \lipsum[A-B], where A and B
                    % are paragraphs numbers (among 47)
\usepackage{blindtext} % \blindtext[3], \Blindtext, \blinddocument,
                       % blindlist{env}[number of items], \blinditemize,
                       % \blindmathtrue, \blindmathpaper
%\usepackage{standalone} % for including compilable documents 
\usepackage{bm}
\usepackage{fp}
\usepackage{minitoc} % for a small table of contents
\usepackage[american,cuteinductors]{circuitikz}
\usepackage{makeidx}
\usepackage{hyperref}
\usepackage{cleveref}
\usepackage[final]{pdfpages} % \includepdf[pages=, ...]{filename}
\usepackage{enumerate}
\usepackage[shortlabels]{enumitem}
\newlist{todolist}{itemize}{2}
\setlist[todolist]{label=$\square$}
\usepackage{setspace}
\usepackage{float}
%%\usepackage{multimedia}
%%\usepackage{media9}
\usepackage{movie15} % for includemovie
%\usepackage{sidenotes} % incompatible with tufte 
%\usepackage{subcaption}  % incompatible with tufte 
%\usepackage[letterpaper]{geometry} % incompatible with tufte
\usepackage{array} % to improve the tables, adds commands like m{}
\usepackage{booktabs} % better tables: \toprule, \midrule, \bottomrule,
                      % \cmidrule, \arrayrulecolor (with colortbl)
\usepackage{multirow} %% for multirow command in tables
\usepackage{colortbl}
\usepackage{longtable}  % for tables on more than one page
\usepackage{multicol}
%\usepackage{microtype}  %% Improves letter spacing % INCOMPATIBLE with esami
\usepackage{listings} %% for source code examples
%\usepackage{MnSymbol} %% for \sumint symbol
\usepackage{bigints} %% for \bigoint symbol
%\usepackage{fancybox}  % fancy border  and shadowed boxed
%\usepackage[Lenny]{fncychap} %% For fancy chapter headings
\usepackage{fancyhdr} % footers and headers % incompatible with memoir
\pagestyle{fancy} % footers and headings % incompatible with memoir
%\fvset{fontsize=\normalsize} % 
\usepackage{wrapfig}
%%\usepackage{psfrag}
%\usepackage{epsfig}
%\usepackage{tocbibind}
%\usepackage[spanish,es-tabla,es-nodecimaldot,es-lcroman]{babel} % incompatible with esami
\usepackage[colorinlistoftodos]{todonotes} % requires tikzexternalizeoptimizefalse
%\usepackage{marginfix} % to fix issues with margin notes

%% SETTINGS FOR SEVERAL PACKAGES
% tikz/pgf
\pgfplotsset{compat=1.16}
\usepgfplotslibrary{external}
\tikzexternalize%[mode=list and make] % sometimes this mode does not
                %work, does not recompute the figures
%\tikzexternalize[/tikz/external/optimize=false] %for todonotes, better use xtodo macro
%\tikzset{external/system call={pdflatex \tikzexternalcheckshellescape -halt-on-error -interaction=batchmode -jobname ``\image'' ``\texsource''}} % to let pdflatex work compile picture: pdflatex --shell-escape xxxxxxx.tex
\usetikzlibrary{shapes,arrows,circuits,calc,babel,arrows.meta,shapes.misc} % babel is useful to avoid collisions with tikz <-> and spanish babel
\tikzset{>=Latex} %% For better default arrows
\tikzset{cross/.style={path picture={ \draw[black,thick]
            (path picture bounding box.south east) --
            (path picture bounding box.north west)
            (path picture bounding box.south west) --
            (path picture bounding box.north east);
}}}
\pgfmathsetseed{25}
\usepgflibrary{fpu}
\usetikzlibrary{fpu}
\pgfkeys{/pgf/fpu}
\pgfset{fpu}
\pgfkeys{/pgf/number format/.cd,
  sci,
  sci e,
  sci zerofill,
  precision = 3,
%  % /pgf/number format/assume math mode=true,
%  %sci generic = {mantissa sep=\times, exponent={10^{#1}}}
}
% siunitx
\sisetup{round-half=even,inter-unit-product=\ensuremath{\cdot},zero-decimal-to-integer,exponent-product=\cdot,output-product=\cdot}
\DeclareSIUnit\mm{\milli\meter}
% \SendSettingsToPgf
% circuitikz
\ctikzset{bipoles/thickness=2}
\ctikzset{bipoles/capacitor/height=0.4}
\tikzset{C/.append style={color=blue, thick, *-*}}
\tikzset{R/.append style={color=red, thick, *-*}}
\tikzset{battery1/.append style={color=teal!80, thick, *-*}}
\tikzset{battery/.append style={color=teal!80, thick, *-*}}
% figurespath
\graphicspath{{fig/}{./}}
\DeclareGraphicsExtensions{.pdf,.pstex,.eps,.png,.jpg,}
% hyperref
\hypersetup{colorlinks}
% makeindex
\makeindex


\hyphenation{sig-ni-fi-ca-ti-vas co-rrien-te lle-va dis-mi-nu-ye lle-van}

% command helpers
\setcounter{secnumdepth}{1}
%%%%%%%%%%%%%%%%%%%%%%%%%%%% 
%% fancy printing for commands, labels, etc 
%% settings for listings package, extracted from http://tex.stackexchange.com/questions/46953/unix-command-highlighting-latex
%%%%%%%%%%%%%%%%%%%%%%%%%%%% 
\lstdefinestyle{BashInputStyle}{
  language=bash,
  basicstyle=\small\sffamily,
  breaklines=true,
  % numbers=left,
  %numberstyle=\tiny,
  %numbersep=3pt,
  frame=tb,
  columns=fullflexible,
  backgroundcolor=\color{yellow!20},
  % linewidth=0.9\linewidth,
  xleftmargin=0.1\linewidth}
\newcommand{\cmd}[1]{\begin{lstlisting}[style=BashInputStyle] #1 \end{lstlisting}}
\newcommand{\ins}[1]{\begin{shaded} \texttt{#1} \end{shaded} } % does not work with special math chars as -- or _
\newcommand{\fname}[1]{\fcolorbox{white}{Dandelion!25}{\texttt{#1}}}
